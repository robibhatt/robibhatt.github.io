% LaTeX file for CV 
% This file uses the resume document class (res.cls)

\documentclass{res} 
\usepackage{hyperref}
\setlength{\textheight}{9.5in} % increase text height to fit more content

\begin{document} 

\name{ROBI BHATTACHARJEE\\[12pt]}

\address{28 Quenstedtstrasse\\ Tübingen 72076, Germany}
\address{\bf CONTACT INFORMATION \\ robibhatt@gmail.com \\ robibhatt.github.io \\ github.com/robibhatt }
                                  
\begin{resume}
         
\section{EDUCATION} 
University of California, San Diego  \\        
\quad PhD in Computer Science, Fall 2018 – July 2023 \\
\quad Advised by Kamalika Chaudhuri and Sanjoy Dasgupta \\[1ex]
Massachusetts Institute of Technology  \\        
\quad Bachelor of Science, General Mathematics, June 2016 \\

\section{WORK EXPERIENCE}
\vspace{-0.1in}
\begin{tabbing}
\hspace{2.3in}\= \hspace{2.6in}\= \kill
{\bf Postdoctoral Researcher} \>University of Tübingen    \>Oct 2023 – Present\\
\>Tübingen, Germany
\end{tabbing}\vspace{-20pt}
Supervised by Ulrike von Luxburg. Conducting research in theoretical machine learning.  
\vspace{-0.1in}

\begin{tabbing}
\hspace{2.3in}\= \hspace{2.6in}\= \kill
{\bf PhD Student, Graduate RA} \>UC San Diego    \>Fall 2018 – July 2023\\
\>San Diego, CA
\end{tabbing}\vspace{-20pt}
Conducted research in adversarial robustness, clustering, and trustworthy ML under Kamalika Chaudhuri and Sanjoy Dasgupta.  
\vspace{-0.1in}

\begin{tabbing}
\hspace{2.3in}\= \hspace{2.6in}\= \kill
{\bf Assistant Trader} \>Five Rings Capital    \>June 2016 – July 2017\\
\>New York, NY
\end{tabbing}\vspace{-20pt}
Developed automated trading strategies; coded simulations on historical market data.  
\vspace{-0.1in}

\begin{tabbing}
\hspace{2.3in}\= \hspace{2.6in}\= \kill
{\bf Research Intern} \>Jane Street Capital    \>Summer 2015\\
\>New York, NY
\end{tabbing}\vspace{-20pt}
Worked on trading strategies and statistical tools for improving them.  
\vspace{-0.1in}

\begin{tabbing}
\hspace{2.3in}\= \hspace{2.6in}\= \kill
{\bf Software Engineering Intern} \>Google    \>Summer 2014\\
\>Mountain View, CA
\end{tabbing}\vspace{-20pt}
Studied the effect of over- and under-sampling to resolve class imbalance for fraud detection.  
\vspace{-0.1in}

\begin{tabbing}
\hspace{2.3in}\= \hspace{2.6in}\= \kill
{\bf Trading/Research Intern} \>Jane Street Capital    \>Summer 2013\\
\>New York, NY
\end{tabbing}\vspace{-20pt}
Worked on trading strategies and participated in mock trading exercises.  
\vspace{-0.1in}

\section{TEACHING}

\textbf{Teaching Assistant, CSE 251B (Graduate Machine Learning)} \\
UC San Diego, Winter 2022 \\
Led discussion sections, held office hours, graded assignments and exams.

\textbf{Teaching Assistant, CSE 251B (Graduate Machine Learning)} \\
UC San Diego, Winter 2021 \\
Led discussion sections, held office hours, graded assignments and exams.

\textbf{Teaching Assistant, Seminar “Explainable Machine Learning”} \\
University of Tübingen, Summer 2024 \\
Co-led by Ulrike von Luxburg and Robi Bhattacharjee. Delivered lectures (e.g. on LIME/SHAP), guided student presentations, and organized the seminar schedule.

\section{SUPERVISION}

\textbf{Clara Groethans}, Master’s Student, University of Tübingen, Oct 2024 – Jun 2025 \\
Thesis: \textit{Understanding the Interplay between Model Complexity and Explanation Quality: Measuring Local Fidelity}. \\[0.5em]

\textbf{Harald Kugler}, Master’s Student, University of Tübingen, Apr 2024 – Oct 2024 \\
Thesis: \textit{Towards a Reliable and Scalable Data-Copying Detection Algorithm Using Random Projections}. \\

\section{PUBLICATIONS}

Robi Bhattacharjee, Geelon So, and Sanjoy Dasgupta. "Consistency of the $k_n$-nearest neighbor rule under adaptive sampling." NeurIPS 2025. \\

Robi Bhattacharjee, Karolin Frohnapfel, and Ulrike von Luxburg. "How to Safely Discard Features Based on Aggregate SHAP Values." COLT 2025. \\

Robi Bhattacharjee and Ulrike von Luxburg. "Auditing Local Explanations is Hard." NeurIPS 2024. \\

Robi Bhattacharjee, Sanjoy Dasgupta, and Kamalika Chaudhuri. "Data-Copying in Generative Models: A Formal Framework." ICML 2023. \\

Robi Bhattacharjee, Max Hopkins, Akash Kumar, Hantao Yu, and Kamalika Chaudhuri. "Robust Empirical Risk Minimization with Tolerance." ALT 2023. \\

Robi Bhattacharjee, Jacob Imola, Michal Moshkovitz, and Sanjoy Dasgupta. "Online $k$-means Clustering on Arbitrary Data Streams." ALT 2023. \\

Robi Bhattacharjee and Gaurav Mahajan. "Learning What to Remember." ALT 2022. \\

Robi Bhattacharjee and Kamalika Chaudhuri. "Consistent Non-Parametric Methods for Maximizing Robustness." NeurIPS 2021. \\

Robi Bhattacharjee, Somesh Jha, and Kamalika Chaudhuri. "Sample Complexity of Adversarially Robust Linear Classification on Separated Data." ICML 2021. \\

Robi Bhattacharjee and Michal Moshkovitz. "No-substitution $k$-means Clustering with Adversarial Order." ALT 2021. \\

Robi Bhattacharjee and Kamalika Chaudhuri. "When are Non-Parametric Methods Robust?" ICML 2020. \\

Robi Bhattacharjee and Sanjoy Dasgupta. "What Relations are Reliably Embeddable in Euclidean Space?" ALT 2020. \\

\section{PREPRINTS / IN SUBMISSION}

Eric Günther, Balázs Szabados, Robi Bhattacharjee, Sebastian Bordt, and Ulrike von Luxburg. "Informative Post-Hoc Explanations Only Exist for Simple Functions." arXiv preprint, 2025. \\

Robi Bhattacharjee, Nick Rittler, and Kamalika Chaudhuri. "Beyond Discrepancy: A Closer Look at the Theory of Distribution Shift." In submission. \\

\section{SERVICE}
Mentor for UCSD Graduate Women in Computing (Fall 2020 – Winter 2021).\\
Geometry teacher for MISE foundation (Winter – Summer 2021).\\
Reviewer for AISTATS (Fall 2020).\\
Reviewer for JMLR (Winter 2021).  

\section{REVIEWING}
Reviewer, NeurIPS (2025, 2024, 2023).\\
Reviewer, ICML (2025, 2024, 2022, 2021).\\
Reviewer, ICLR (2025).\\
Reviewer, ALT (2025).\\

\section{HONORS AND AWARDS}
Honorable Mention on the William Lowell Putnam Examination 2012, 2013.\\
9th place individual nationally in American Regional Mathematics Competition.\\
Participant of the Math Olympiad Summer Program, 2010.\\
USAMO qualifier 2009–2012 (ranked 26th in nation in 2012).  

\section{TALKS}
"How to Safely Discard Features Based on Aggregate SHAP Values." COLT 2025. \\

"Robust Empirical Risk Minimization with Tolerance." ALT 2023. \\

"Online $k$-means Clustering on Arbitrary Data Streams." ALT 2023. \\

"Learning What to Remember." ALT 2022. \\

"No-substitution $k$-means Clustering with Adversarial Order." ALT 2021. \\

"When are Non-Parametric Methods Robust?" ICML 2020. \\

"What Relations are Reliably Embeddable in Euclidean Space?" ALT 2020. \\

"No-substitution $k$-means Clustering with Adversarial Order." Algorithmic Learning Theory, Spring 2021. \\

"No-substitution $k$-means Clustering with Adversarial Order." UCSD Theory Group, Winter 2021. \\

"When are Non-Parametric Methods Robust?" International Conference on Machine Learning, Summer 2020. \\

"What Relations are Reliably Embeddable in Euclidean Space?" Algorithmic Learning Theory, Winter 2020. \\

"Basics of Information Theory." San Diego Math Circle, Fall 2020. \\

"What Relations are Reliably Embeddable in Euclidean Space?" SoCalML, Spring 2019. \\

\end{resume}
\end{document}
